\chapter{Couchbase}

\begin{wrapfigure}{l}{0.3\textwidth}
  \vspace{-100pt}
  \begin{center}
    \includegraphics[width=0.28\textwidth]{couchbase.png}
  \end{center}
  \vspace{-30pt}
\end{wrapfigure}
Couchbase Server is a distributed, document database management system, designed to store the information for web applications. Couchbase Server primarily uses RAM as the storage mechanism, enabling it to support very fast create, store, update and retrieval operations.

These features are designed to support web application development where the high-performance characteristics are required to support low-latency and high throughput applications. Couchbase Server achieves this on a single server and provides support for the load to be increased almost linearly by making use of the clustered functionality built into Couchbase Server.

\section{Short specification}

\begin{itemize}
  \item \textbf{Written in:} C++, Erlang
  \item \textbf{Main point:} Memcache compatible. With persistence, clustering and documents storage
  \item \textbf{License:} Apache License 2.0
  \item \textbf{Protocol:} HTTP/REST, memcache protocol
  \item \textbf{Web site:} \href{http://www.couchbase.com/}{www.couchbase.com}
\end{itemize}

\section{Main features}

\subsection{Memcached compatible}

With memcached inside, Couchbase Server enables applications to read and write data with sub-millisecond latency and sustained high throughput. Efficient memory management delivers predictable latency: an object-level cache prevents thrashing inherent to page-level approaches. Users can allocate cache resources per database depending on application needs, rather than relying on global cache management by the operating system.

\subsection{Zero-downtime maintenance}

Couchbase Server allows you to perform any maintenance tasks on a live cluster. Not only do cluster topology changes like adding or removing servers require no application downtime, but even upgrading software can be done on a running system.

\section{Strengths}

\section{Weaknesses}

\section{Tips}

\section{Use cases}

\begin{itemize}
  \item Cloud infrastructure
  \item Scalable, low-latency storage for mobile apps
  \item Targeting
  \item Highly-concurrent web apps like online gaming
\end{itemize}