\chapter{Redis}

\begin{wrapfigure}{l}{0.3\textwidth}
  \vspace{-80pt}
  \begin{center}
    \includegraphics[width=0.28\textwidth]{redis.png}
  \end{center}
  \vspace{-20pt}
\end{wrapfigure}
Redis is an open source, advanced key-value store. It is often referred to as a data structure server since keys can contain strings, hashes, lists, sets and sorted sets.
You can run atomic operations on these types, like appending to a string; incrementing the value in a hash; pushing to a list; computing set intersection, union and difference; or getting the member with highest ranking in a sorted set.

\section{Short specification}

\begin{itemize}
  \item \textbf{Written in:} C/C++ (ANSI C)
  \item \textbf{Main point:} Blazing fast
  \item \textbf{License:} BSD
  \item \textbf{Protocol:} Telnet-like
  \item \textbf{Web site:} \href{http://redis.io}{redis.io}
\end{itemize}

\section{Main features}

\section{Strengths}

The obvious strength of Redis is speed, like so many key-value stores of its ilk. But more than most key-value stores, Redis provides the ability to store complex values like lists, hashes, and sets, and retrieve them based through operations specific to those datatypes. Beyond even a data structure store, however, Redis's durability options allow you to trade speed for data safety up to a fairly fine point. Built-in master-slave replication is another nice way of ensuring better durability without requiring the slowness of syncing an append-only file to disk on every operation. Additionally, replication is great for very high-read systems.\cite{seven_databases}

\section{Weaknesses}

Redis is fast largely because it resides in memory. Some may consider this cheating, since of course a database that never hits the disk will be fast. A main memory database has an inherent durability problem; namely, if you shut down the database before a snapshot occurs, you can lose data. Even if you set the append-only file to disk sync on every operation, you run a risk with playing back expiry values, since time-based events can never be counted on to replay in exactly the same manner—though in fairness this case is more hypothetical than practical.

Redis also does not support datasets larger than your available RAM (Redis is removing virtual memory support), so its size has a practical limitation. Although there is a Redis Cluster currently in development to grow beyond a single-machine’s RAM requirements, anyone wanting to cluster Redis must currently roll their own with a client that supports it.\cite{seven_databases}

\section{Tips}

\section{Use cases}

\begin{itemize}
  \item Session Storage
  \item Cache Storage
  \item Job Queue
  \item Real time analysis
  \item Pub/Sub
\end{itemize}