\chapter{FlockDB}

\begin{wrapfigure}{l}{0.3\textwidth}
  \vspace{-75pt}
  \begin{center}
    \includegraphics[width=0.28\textwidth]{flockdb.png}
  \end{center}
  \vspace{-30pt}
\end{wrapfigure}
FlockDB is an open source distributed, fault-tolerant graph database for managing data at webscale. It was initially used by Twitter to build its database of users and manage their relationships to one another. The developers claim that it is much simpler than other graph databases since it scales horizontally and is designed for on-line, low-latency, high throughput environments such as websites. Since it is still in the process of being packaged for outside of Twitter use, the code is still very rough and hence there is no stable release available yet. FlockDB was posted on GitHub shortly after Twitter released its Gizzard framework, which it uses to query the FlockDB distributed datastore. 

\section{Short specification}

\begin{itemize}
  \item \textbf{Written in:} Scala, Java, Ruby
  \item \textbf{Main point:} Fault tolerant graph database
  \item \textbf{License:} Apache License 2
  \item \textbf{Protocol:} Custom binary
  \item \textbf{Web site:} \href{https://github.com/twitter/flockdb}{github.com/twitter/flockdb}
\end{itemize}

\section{Main features}

\section{Strengths}

\section{Weaknesses}

\section{Tips}

\section{Use cases}
