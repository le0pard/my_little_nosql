\chapter{HBase}

HBase is an open source, non-relational, distributed database modeled after Google's BigTable and is written in Java. It is developed as part of Apache Software Foundation's Apache Hadoop project and runs on top of HDFS (Hadoop Distributed Filesystem), providing BigTable-like capabilities for Hadoop. That is, it provides a fault-tolerant way of storing large quantities of sparse data.

HBase features compression, in-memory operation, and Bloom filters on a per-column basis as outlined in the original BigTable paper. Tables in HBase can serve as the input and output for MapReduce jobs run in Hadoop, and may be accessed through the Java API but also through REST, Avro or Thrift gateway APIs.

\section{Short specification}

\begin{itemize}
  \item \textbf{Written in:} Java
  \item \textbf{Main point:} Billions of rows X millions of columns
  \item \textbf{License:} Apache License 2.0
  \item \textbf{Protocol:} HTTP/REST (also Thrift)
  \item \textbf{Web site:} \href{http://hbase.apache.org/}{hbase.apache.org}
\end{itemize}